%%%%%%%%%%%%%%%%%%%%%%%%%%%%%%%%%%%%%%%%%
% Beamer Presentation
% LaTeX Template
% Version 1.0 (10/11/12)
%
% This template has been downloaded from:
% http://www.LaTeXTemplates.com
%
% License:
% CC BY-NC-SA 3.0 (http://creativecommons.org/licenses/by-nc-sa/3.0/)
%
%%%%%%%%%%%%%%%%%%%%%%%%%%%%%%%%%%%%%%%%%

%----------------------------------------------------------------------------------------
%	PACKAGES AND THEMES
%----------------------------------------------------------------------------------------

\documentclass{beamer}

\mode<presentation> {

% The Beamer class comes with a number of default slide themes
% which change the colors and layouts of slides. Below this is a list
% of all the themes, uncomment each in turn to see what they look like.

%\usetheme{default}
%\usetheme{AnnArbor}
%\usetheme{Antibes}
%\usetheme{Bergen}
%\usetheme{Berkeley}
%\usetheme{Berlin}
%\usetheme{Boadilla}
%\usetheme{CambridgeUS}
%\usetheme{Copenhagen}
%\usetheme{Darmstadt}
%\usetheme{Dresden}
%\usetheme{Frankfurt}
%\usetheme{Goettingen}
%\usetheme{Hannover}
%\usetheme{Ilmenau}
%\usetheme{JuanLesPins}
%\usetheme{Luebeck}
\usetheme{Madrid}
%\usetheme{Malmoe}
%\usetheme{Marburg}
%\usetheme{Montpellier}
%\usetheme{PaloAlto}
%\usetheme{Pittsburgh}
%\usetheme{Rochester}
%\usetheme{Singapore}
%\usetheme{Szeged}
%\usetheme{Warsaw}

% As well as themes, the Beamer class has a number of color themes
% for any slide theme. Uncomment each of these in turn to see how it
% changes the colors of your current slide theme.

%\usecolortheme{albatross}
%\usecolortheme{beaver}
%\usecolortheme{beetle}
%\usecolortheme{crane}
%\usecolortheme{dolphin}
%\usecolortheme{dove}
%\usecolortheme{fly}
%\usecolortheme{lily}
%\usecolortheme{orchid}
%\usecolortheme{rose}
%\usecolortheme{seagull}
%\usecolortheme{seahorse}
%\usecolortheme{whale}
%\usecolortheme{wolverine}

%\setbeamertemplate{footline} % To remove the footer line in all slides uncomment this line
%\setbeamertemplate{footline}[page number] % To replace the footer line in all slides with a simple slide count uncomment this line

%\setbeamertemplate{navigation symbols}{} % To remove the navigation symbols from the bottom of all slides uncomment this line
}

\usepackage{graphicx} % Allows including images
\usepackage{booktabs} % Allows the use of \toprule, \midrule and \bottomrule in tables
\usepackage[mathscr]{eucal}
\usepackage{latexsym}
\usepackage{hyperref}
\usepackage{amssymb}
\usepackage{psfrag}
\usepackage{epsfig}
\usepackage{enumerate}
\usepackage{appendixnumberbeamer}
\DeclareMathAlphabet{\mathpzc}{OT1}{pzc}{m}{it}
\usepackage{wrapfig}
\usepackage{xcolor}

\include{psbox}

\usepackage{amsfonts}
\usepackage{graphicx}
\usepackage{amsfonts}
\usepackage[textsize=small]{todonotes}


\DeclareMathOperator*{\argmin}{arg\,min}
\DeclareMathOperator*{\argmax}{arg\,max}
\DeclareMathOperator*{\essinf}{ess\,inf}
\DeclareMathOperator*{\esssup}{ess\,sup}
\DeclareMathOperator*{\Tr}{Tr}

\newcommand{\AB}[1]{\todo[color=NavyBlue,inline]{Amarjit: #1}}
\newcommand{\EF}[1]{\todo[color=Orange,inline]{Eric: #1}}


\usepackage[numbers]{natbib}


\begin{document}





\newcommand{\skp}{\vspace{\baselineskip}}
\newcommand{\noi}{\noindent}
\newcommand{\osc}{\mbox{osc}}
\newcommand{\lfl}{\lfloor}
\newcommand{\rfl}{\rfloor}
\newcommand{\img}{\imath}
\newcommand{\iy}{\infty}
\newcommand{\eps}{\varepsilon}
\newcommand{\del}{\delta}
\newcommand{\Rk}{\mathbb{R}^k}
\newcommand{\RR}{\mathbb{R}}
\newcommand{\spa}{\vspace{.2in}}
\newcommand{\V}{\mathcal{V}}
\newcommand{\E}{\mathbb{E}}
\newcommand{\I}{\mathbb{I}}
%\newcommand{\P}{\mathbb{P}}
\newcommand{\PP}{\mathbb{P}}
\newcommand{\sgn}{\mbox{sgn}}
\newcommand{\ti}{\tilde}

\newcommand{\QQ}{\mathbb{Q}}

\newcommand{\XX}{\mathbb{X}}
\newcommand{\XXz}{\mathbb{X}^0}

\newcommand{\lan}{\langle}
\newcommand{\ran}{\rangle}
\newcommand{\lf}{\lfloor}
\newcommand{\rf}{\rfloor}
\def\wh{\widehat}
\newcommand{\defn}{\stackrel{def}{=}}
\newcommand{\txb}{\tau^{\epsilon,x}_{B^c}}
\newcommand{\tyb}{\tau^{\epsilon,y}_{B^c}}
\newcommand{\tilxb}{\tilde{\tau}^\eps_1}
\newcommand{\pxeps}{\mathbb{P}_x^{\eps}}
\newcommand{\non}{\nonumber}
\newcommand{\dist}{\mbox{dist}}


\newcommand{\Om}{\mathnormal{\Omega}}
\newcommand{\om}{\omega}
\newcommand{\vph}{\varphi}
\newcommand{\Del}{\mathnormal{\Delta}}
\newcommand{\Gam}{\mathnormal{\Gamma}}
\newcommand{\Sig}{\mathnormal{\Sigma}}

\newcommand{\tilyb}{\tilde{\tau}^\eps_2}
\newcommand{\beq}{\begin{eqnarray*}}
\newcommand{\eeq}{\end{eqnarray*}}
\newcommand{\beqn}{\begin{eqnarray}}
\newcommand{\eeqn}{\end{eqnarray}}
\newcommand{\ink}{\rule{.5\baselineskip}{.55\baselineskip}}

\newcommand{\bt}{\begin{theorem}}
\newcommand{\et}{\end{theorem}}
\newcommand{\deps}{\Del_{\eps}}
\newcommand{\dbl}{\mathbf{d}_{\tiny{\mbox{BL}}}}

\newcommand{\be}{\begin{equation}}
\newcommand{\ee}{\end{equation}}
\newcommand{\ac}{\mbox{AC}}
%\newcommand{\dist}{\mbox{dist}}
\newcommand{\BB}{\mathbb{B}}
\newcommand{\VV}{\mathbb{V}}
\newcommand{\DD}{\mathbb{D}}
\newcommand{\KK}{\mathbb{K}}
\newcommand{\HH}{\mathbb{H}}
\newcommand{\TT}{\mathbb{T}}
\newcommand{\CC}{\mathbb{C}}
\newcommand{\ZZ}{\mathbb{Z}}
\newcommand{\SSS}{\mathbb{S}}
\newcommand{\EE}{\mathbb{E}}
\newcommand{\NN}{\mathbb{N}}
\newcommand{\MM}{\mathbb{M}}


\newcommand{\clg}{\mathcal{G}}
\newcommand{\clb}{\mathcal{B}}
\newcommand{\cls}{\mathcal{S}}
\newcommand{\clc}{\mathcal{C}}
\newcommand{\clj}{\mathcal{J}}
\newcommand{\clm}{\mathcal{M}}
\newcommand{\clx}{\mathcal{X}}
\newcommand{\cld}{\mathcal{D}}
\newcommand{\cle}{\mathcal{E}}
\newcommand{\clv}{\mathcal{V}}
\newcommand{\clu}{\mathcal{U}}
\newcommand{\clr}{\mathcal{R}}
\newcommand{\clt}{\mathcal{T}}
\newcommand{\cll}{\mathcal{L}}
\newcommand{\clz}{\mathcal{Z}}
\newcommand{\clq}{\mathcal{Q}}

\newcommand{\cli}{\mathcal{I}}
\newcommand{\clp}{\mathcal{P}}
\newcommand{\cla}{\mathcal{A}}
\newcommand{\clf}{\mathcal{F}}
\newcommand{\clh}{\mathcal{H}}
\newcommand{\clo}{\mathcal{O}}
\newcommand{\N}{\mathbb{N}}
\newcommand{\Q}{\mathbb{Q}}
\newcommand{\bfx}{{\boldsymbol{x}}}
\newcommand{\bfa}{{\boldsymbol{a}}}
\newcommand{\bfh}{{\boldsymbol{h}}}
\newcommand{\bfs}{{\boldsymbol{s}}}
\newcommand{\bfm}{{\boldsymbol{m}}}
\newcommand{\bff}{{\boldsymbol{f}}}
\newcommand{\bfb}{{\boldsymbol{b}}}
\newcommand{\bfw}{{\boldsymbol{w}}}
\newcommand{\bfz}{{\boldsymbol{z}}}
\newcommand{\bfu}{{\boldsymbol{u}}}
\newcommand{\bfell}{{\boldsymbol{\ell}}}
\newcommand{\bfn}{{\boldsymbol{n}}}
\newcommand{\bfd}{{\boldsymbol{d}}}
\newcommand{\bfbeta}{{\boldsymbol{\beta}}}
\newcommand{\bfzeta}{{\boldsymbol{\zeta}}}
\newcommand{\bfnu}{{\boldsymbol{\nu}}}
\newcommand{\bfvarphi}{{\boldsymbol{\varphi}}}

\newcommand{\curvz}{{\bf \mathpzc{z}}}
\newcommand{\curvx}{{\bf \mathpzc{x}}}
\newcommand{\curvi}{{\bf \mathpzc{i}}}
\newcommand{\curvs}{{\bf \mathpzc{s}}}
\newcommand{\blip}{\mathbb{B}_1}

\newcommand{\BM}{\mbox{BM}}

\newcommand{\tac}{\mbox{\scriptsize{AC}}}
%----------------------------------------------------------------------------------------
%	TITLE PAGE
%----------------------------------------------------------------------------------------

\title[Triage]{Simulating a Mass-Casualty Event} % The short title appears at the bottom of every slide, the full title is only on the title page

% \author[Eric Friedlander]{Eric Friedlander} % Your name
% \institute[UNC-CH] % Your institution as it will appear on the bottom of every slide, may be shorthand to save space
% {
% University of North Carolina at Chapel Hill\\ % Your institution for the title page
% Department of Statistics and Operations Research
% }
\date{May 21st, 2019} % Date, can be changed to a custom date

\begin{frame}
\titlepage % Print the title page as the first slide
\end{frame}

%----------------------------------------------------------------------------------------
%	PRESENTATION SLIDES
%----------------------------------------------------------------------------------------

%
% \section{Introduction}
% \frame{\tableofcontents [
%         currentsection,
%         currentsubsection,
%         subsectionstyle=show/shaded/hide
%     ]}


\begin{frame}
\frametitle{Mass-Casualty Event}
\includegraphics[width = .49\textwidth, height = .3\textwidth]{earthquake}
\includegraphics[width = .49\textwidth, height = .3\textwidth]{busaccident}
\includegraphics[width = .49\textwidth, height = .3\textwidth]{tsunami}
\includegraphics[width = .49\textwidth, height = .3\textwidth]{bombing}
\end{frame}

\begin{frame}
  \frametitle{Triage}
  \begin{minipage}[c]{.45\textwidth}
    \includegraphics[width=1\textwidth]{EMSTriage}
  \end{minipage}
  \begin{minipage}[c]{.45\textwidth}
    \begin{itemize}
      \item \textbf{First-responders:} Triage patients based on injury severity
      \item \textcolor{red}{RED:} Patients in worst shape
      \item \textcolor{yellow}{YELLOW:} Patients whose care can be delayed
      \item \textcolor{green}{GREEN:} Patients with minor injuries
    \end{itemize}
  \end{minipage}
\end{frame}

\begin{frame}
  \frametitle{Simple Triage and Rapid Treatment (START)}
  \centering
  \includegraphics[width=1\textwidth]{Start}
\end{frame}

\begin{frame}
  \frametitle{Simple Triage and Rapid Treatment (START)}
  \centering
  \includegraphics[width=.5\textwidth]{StartAdultTriageAlgorithm}
\end{frame}

\begin{frame}
  \frametitle{Oh no! The red line has derailed!}
  \includegraphics[width=1\textwidth]{train_derailed}
\end{frame}


\begin{frame}
  \frametitle{Meanwhile, you must prepare the UChicago Trauma Unit}
  \includegraphics[width=1\textwidth]{uchicago_trama}
\end{frame}

\begin{frame}
  \frametitle{Houston, we have a problem...}
  \begin{itemize}
    \item \textbf{Problem:} Not enough resources for all injured
    \item Expect around 30 patients but we only have the capacity to help 10!\pause
    \item Any patients who cannot be seen here will need to be transported to the next closest unit which will take an additional 50 minutes!\pause
    \item \textbf{Who should be admitted and who should we turn away?}
  \end{itemize}
\end{frame}

\begin{frame}
  \frametitle{What are some policies you might consider?}
  \pause
  Only admit IMMEDIATE or DELAYED patients
  \pause
  \begin{itemize}
    \item First-Come-First-Serve
    \item Reserve beds for IMMEDIATE patients
    \item Reserve beds for DELAYED patients
    \item Reserve some for IMMEDIATE and some for DELAYED
    \item Other?
  \end{itemize}
\end{frame}

\begin{frame}
  \frametitle{What information might help you decide?}
  \pause
  \begin{itemize}
    \item How are patients arriving?
    \begin{itemize}
      \item[-] How fast do they arrive?
      \item[-] When do they arrive?
      \item[-] How many IMMEDIATE vs. DELAYED?
    \end{itemize}
  \end{itemize}
  \pause
  \begin{itemize}
    \item Will patients survive the additional time it takes to get to another hospital?
    \begin{itemize}
      \item[-] What is their probability of survival?
      \item[-] How does this change with time?
      \item[-] How are these different for IMMEDIATE vs. DELAYED patients?
    \end{itemize}
    \pause
    \item Let's use computers to help us!
  \end{itemize}
\end{frame}

\begin{frame}
  \frametitle{Based On Actual Research!}
  \centering
  \includegraphics[scale = .3]{paper}
\end{frame}


% \begin{frame}
%   \frametitle{Humans are very predictable!}
%   \begin{itemize}
%     \item Many of the things discussed above are random
%     \item Why might we want to use a computer to generate random events rather than just doing it ourself? \pause
%     \begin{itemize}
%       \item Human's are very bad at being truly random
%       \item \textcolor{blue}{\href{http://www.mindreaderpro.appspot.com/}{Mindreader!}}
%     \end{itemize}
%   \end{itemize}
% \end{frame}

% \begin{frame}
%   \frametitle{Modeling Arrivals}
% \end{frame}
%
%
% \begin{frame}
%   \frametitle{Modeling Survival}
% \end{frame}
%
%
%
% \begin{frame}
%   \frametitle{SIMULATE!}
% \end{frame}


% \begin{frame}
%   \frametitle{How can you learn more?}
%   There are \textbf{TONS} of \textbf{FREE RESOURCES}:
%   \begin{itemize}
%     \item Google python tutorial: https://developers.google.com/edu/python/
%     \item MIT Open Courseware: https://ocw.mit.edu/courses/electrical-engineering-and-computer-science/6-00sc-introduction-to-computer-science-and-programming-spring-2011/
%     \item Harvard online courses: https://online-learning.harvard.edu/course/cs50-introduction-computer-science
%     \item Codecademy: https://www.codecademy.com/
%     \item Hackerrank: https://www.hackerrank.com/
%   \end{itemize}
% \end{frame}





\end{document}
